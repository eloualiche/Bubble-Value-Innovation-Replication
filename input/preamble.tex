
%%%%%%%%%%%%%%%%%%%%%%%%%%%%
% SET foNt chOicES
\usepackage{eulervm}

\usepackage[T1]{fontenc}
\usepackage{inconsolata}

% If you go the XeTeX Route
% \usepackage{xltxtra}
% \usepackage{fontspec} %Font package
% \usepackage{xunicode}
% \setmonofont{Operator Mono Light}
% \setmonofont{Hasklig}


%%%%%%%%%%%%%%%%%%%%%%%%%%%%
\usepackage{amsmath}
\usepackage{amssymb}
\usepackage{mathtools}
\usepackage{bm}
\usepackage{empheq}
\usepackage{xcolor}
\usepackage{geometry}
\usepackage[page,header]{appendix}
\usepackage{titletoc}
\usepackage{eufrak}
\usepackage{pdflscape}
\usepackage{lscape}
\usepackage{afterpage}
\usepackage{floatpag}

% ------------------------------------------------------------------------------------
\usepackage{graphicx} % allow embedded images
  \setkeys{Gin}{width=\linewidth,totalheight=\textheight,keepaspectratio}
  \graphicspath{{tmp/}} % set of paths to search for images
\usepackage{booktabs} % book-quality tables
\usepackage{xfrac}    % for \sfrac macro
\usepackage{units}    % non-stacked fractions and better unit spacing
\usepackage{multicol} % multiple column layout facilities
\usepackage{numprint}
% \usepackage{fancyvrb} % extended verbatim environments
  % \fvset{fontsize=\normalsize}% default font size for fancy-verbatim environments
\usepackage{enumitem}

% ------------------------------------------------------------------------------------
% REQUIRED by huxtable
\usepackage{longtable}
\usepackage{caption}
\usepackage{array}
\usepackage{caption}
\usepackage{graphicx}
\usepackage{siunitx}
\usepackage{colortbl}
\usepackage{multirow}
\usepackage{hhline}
\usepackage{calc}
\usepackage{tabularx}
\usepackage{wrapfig}
\usepackage[online]{threeparttable}
\usepackage{dcolumn}
\usepackage{bbding}

% ------------------------------------------------------------------------------------


% ------------------------------------------------------------------------------------
% \usepackage{subfigure}
\usepackage{subcaption}
\usepackage{comment}
% ------------------------------------------------------------------------------------



% ------------------------------------------------------------------------------------
\PassOptionsToPackage{table,usenames,dvipsnames*,svgnames,hyperref}{xcolor}
\RequirePackage{xcolor}
\definecolor{light-gray}{gray}{0.85}
\definecolor{dark-gray}{gray}{0.35}
\colorlet{halfmaroon}{red!55!black}
\definecolor{maroon}{RGB}{128, 0, 0}
\usepackage[colorlinks,linktoc=page]{hyperref}

% Metadata
\hypersetup{pdfauthor={Loualiche},% 
            pdftitle={Asset Pricing with Entry and Imperfect Competition},%
            pdfkeywords={Economics, Finance},%
            pdfsubject={Economics, Finance},%
            pdfcreator={pdfLaTeX},%
            citecolor=dark-gray,%
            urlcolor=maroon,%
            linkcolor=maroon,
            filecolor=dark-gray}
\setcounter{tocdepth}{2}

\newcommand*\mygreybox[1]{%
\colorbox{light-gray}{\hspace{2em}#1\hspace{2em}}}
% ------------------------------------------------------------------------------------



% ------------------------------------------------------------------------------------
% http://www.latex-community.org/forum/viewtopic.php?f=45&t=3970
% proble with nested input in tables: use a tex command within latex
% \newcommand*\ExpandableInput[1]{\@@input #1 }
\makeatletter\let\ExpandableInput\@@input\makeatother 
% \def\input#1{\@@input #1 }


\newcommand\fnote[1]{\captionsetup{font=footnotesize,labelfont=normal,justification=normal}\caption*{#1}}
\newcolumntype{L}{D{.}{.}{2,5}} % for dcolumn package
\newcolumntype{d}{D{.}{.}{-1}}
\newcolumntype{Y}{D..{6.3}}
% ------------------------------------------------------------------------------------



% ------------------------------------------------------------------------------------
% % tikz option s
% \usepackage{tikz}
% \usepackage{pgfplots}
% \pgfplotsset{compat=1.16}
% \usepackage{pgfpages}
% \usepackage{tkz-tab, relsize, etoolbox}
% \usetikzlibrary{arrows.meta, shadows, arrows}

%%%%%%%%%%%%%%%%%%%%%%%%%%%%
% DEfine short cuts
\newcommand\mc{\operatorname{mc}}
% \newcommand\m{\operatorname{\mathfrak{m}}} % \markup
\newcommand{\FCF}{\operatorname{FC}}
\newcommand{\MPL}{\operatorname{MPL}}
\newcommand{\la}{\boldsymbol{\left\langle}}
\newcommand{\ra}{\boldsymbol{\right\rangle}}
\newcommand{\esper}{\mathbf{E}}
\newcommand{\overbar}[1]{\mkern 1.5mu\overline{\mkern-1.5mu#1\mkern-1.5mu}\mkern 1.5mu}
\newcommand{\tnum}[1]{\texttt{\textcolor{maroon}{#1}}}
% ------------------------------------------------------------------------------------



